
% Default to the notebook output style

    


% Inherit from the specified cell style.




    
\documentclass[11pt]{article}

    
    
    \usepackage[T1]{fontenc}
    % Nicer default font (+ math font) than Computer Modern for most use cases
    \usepackage{mathpazo}

    % Basic figure setup, for now with no caption control since it's done
    % automatically by Pandoc (which extracts ![](path) syntax from Markdown).
    \usepackage{graphicx}
    % We will generate all images so they have a width \maxwidth. This means
    % that they will get their normal width if they fit onto the page, but
    % are scaled down if they would overflow the margins.
    \makeatletter
    \def\maxwidth{\ifdim\Gin@nat@width>\linewidth\linewidth
    \else\Gin@nat@width\fi}
    \makeatother
    \let\Oldincludegraphics\includegraphics
    % Set max figure width to be 80% of text width, for now hardcoded.
    \renewcommand{\includegraphics}[1]{\Oldincludegraphics[width=.8\maxwidth]{#1}}
    % Ensure that by default, figures have no caption (until we provide a
    % proper Figure object with a Caption API and a way to capture that
    % in the conversion process - todo).
    \usepackage{caption}
    \DeclareCaptionLabelFormat{nolabel}{}
    \captionsetup{labelformat=nolabel}

    \usepackage{adjustbox} % Used to constrain images to a maximum size 
    \usepackage{xcolor} % Allow colors to be defined
    \usepackage{enumerate} % Needed for markdown enumerations to work
    \usepackage{geometry} % Used to adjust the document margins
    \usepackage{amsmath} % Equations
    \usepackage{amssymb} % Equations
    \usepackage{textcomp} % defines textquotesingle
    % Hack from http://tex.stackexchange.com/a/47451/13684:
    \AtBeginDocument{%
        \def\PYZsq{\textquotesingle}% Upright quotes in Pygmentized code
    }
    \usepackage{upquote} % Upright quotes for verbatim code
    \usepackage{eurosym} % defines \euro
    \usepackage[mathletters]{ucs} % Extended unicode (utf-8) support
    \usepackage[utf8x]{inputenc} % Allow utf-8 characters in the tex document
    \usepackage{fancyvrb} % verbatim replacement that allows latex
    \usepackage{grffile} % extends the file name processing of package graphics 
                         % to support a larger range 
    % The hyperref package gives us a pdf with properly built
    % internal navigation ('pdf bookmarks' for the table of contents,
    % internal cross-reference links, web links for URLs, etc.)
    \usepackage{hyperref}
    \usepackage{longtable} % longtable support required by pandoc >1.10
    \usepackage{booktabs}  % table support for pandoc > 1.12.2
    \usepackage[inline]{enumitem} % IRkernel/repr support (it uses the enumerate* environment)
    \usepackage[normalem]{ulem} % ulem is needed to support strikethroughs (\sout)
                                % normalem makes italics be italics, not underlines
    

    
    
    % Colors for the hyperref package
    \definecolor{urlcolor}{rgb}{0,.145,.698}
    \definecolor{linkcolor}{rgb}{.71,0.21,0.01}
    \definecolor{citecolor}{rgb}{.12,.54,.11}

    % ANSI colors
    \definecolor{ansi-black}{HTML}{3E424D}
    \definecolor{ansi-black-intense}{HTML}{282C36}
    \definecolor{ansi-red}{HTML}{E75C58}
    \definecolor{ansi-red-intense}{HTML}{B22B31}
    \definecolor{ansi-green}{HTML}{00A250}
    \definecolor{ansi-green-intense}{HTML}{007427}
    \definecolor{ansi-yellow}{HTML}{DDB62B}
    \definecolor{ansi-yellow-intense}{HTML}{B27D12}
    \definecolor{ansi-blue}{HTML}{208FFB}
    \definecolor{ansi-blue-intense}{HTML}{0065CA}
    \definecolor{ansi-magenta}{HTML}{D160C4}
    \definecolor{ansi-magenta-intense}{HTML}{A03196}
    \definecolor{ansi-cyan}{HTML}{60C6C8}
    \definecolor{ansi-cyan-intense}{HTML}{258F8F}
    \definecolor{ansi-white}{HTML}{C5C1B4}
    \definecolor{ansi-white-intense}{HTML}{A1A6B2}

    % commands and environments needed by pandoc snippets
    % extracted from the output of `pandoc -s`
    \providecommand{\tightlist}{%
      \setlength{\itemsep}{0pt}\setlength{\parskip}{0pt}}
    \DefineVerbatimEnvironment{Highlighting}{Verbatim}{commandchars=\\\{\}}
    % Add ',fontsize=\small' for more characters per line
    \newenvironment{Shaded}{}{}
    \newcommand{\KeywordTok}[1]{\textcolor[rgb]{0.00,0.44,0.13}{\textbf{{#1}}}}
    \newcommand{\DataTypeTok}[1]{\textcolor[rgb]{0.56,0.13,0.00}{{#1}}}
    \newcommand{\DecValTok}[1]{\textcolor[rgb]{0.25,0.63,0.44}{{#1}}}
    \newcommand{\BaseNTok}[1]{\textcolor[rgb]{0.25,0.63,0.44}{{#1}}}
    \newcommand{\FloatTok}[1]{\textcolor[rgb]{0.25,0.63,0.44}{{#1}}}
    \newcommand{\CharTok}[1]{\textcolor[rgb]{0.25,0.44,0.63}{{#1}}}
    \newcommand{\StringTok}[1]{\textcolor[rgb]{0.25,0.44,0.63}{{#1}}}
    \newcommand{\CommentTok}[1]{\textcolor[rgb]{0.38,0.63,0.69}{\textit{{#1}}}}
    \newcommand{\OtherTok}[1]{\textcolor[rgb]{0.00,0.44,0.13}{{#1}}}
    \newcommand{\AlertTok}[1]{\textcolor[rgb]{1.00,0.00,0.00}{\textbf{{#1}}}}
    \newcommand{\FunctionTok}[1]{\textcolor[rgb]{0.02,0.16,0.49}{{#1}}}
    \newcommand{\RegionMarkerTok}[1]{{#1}}
    \newcommand{\ErrorTok}[1]{\textcolor[rgb]{1.00,0.00,0.00}{\textbf{{#1}}}}
    \newcommand{\NormalTok}[1]{{#1}}
    
    % Additional commands for more recent versions of Pandoc
    \newcommand{\ConstantTok}[1]{\textcolor[rgb]{0.53,0.00,0.00}{{#1}}}
    \newcommand{\SpecialCharTok}[1]{\textcolor[rgb]{0.25,0.44,0.63}{{#1}}}
    \newcommand{\VerbatimStringTok}[1]{\textcolor[rgb]{0.25,0.44,0.63}{{#1}}}
    \newcommand{\SpecialStringTok}[1]{\textcolor[rgb]{0.73,0.40,0.53}{{#1}}}
    \newcommand{\ImportTok}[1]{{#1}}
    \newcommand{\DocumentationTok}[1]{\textcolor[rgb]{0.73,0.13,0.13}{\textit{{#1}}}}
    \newcommand{\AnnotationTok}[1]{\textcolor[rgb]{0.38,0.63,0.69}{\textbf{\textit{{#1}}}}}
    \newcommand{\CommentVarTok}[1]{\textcolor[rgb]{0.38,0.63,0.69}{\textbf{\textit{{#1}}}}}
    \newcommand{\VariableTok}[1]{\textcolor[rgb]{0.10,0.09,0.49}{{#1}}}
    \newcommand{\ControlFlowTok}[1]{\textcolor[rgb]{0.00,0.44,0.13}{\textbf{{#1}}}}
    \newcommand{\OperatorTok}[1]{\textcolor[rgb]{0.40,0.40,0.40}{{#1}}}
    \newcommand{\BuiltInTok}[1]{{#1}}
    \newcommand{\ExtensionTok}[1]{{#1}}
    \newcommand{\PreprocessorTok}[1]{\textcolor[rgb]{0.74,0.48,0.00}{{#1}}}
    \newcommand{\AttributeTok}[1]{\textcolor[rgb]{0.49,0.56,0.16}{{#1}}}
    \newcommand{\InformationTok}[1]{\textcolor[rgb]{0.38,0.63,0.69}{\textbf{\textit{{#1}}}}}
    \newcommand{\WarningTok}[1]{\textcolor[rgb]{0.38,0.63,0.69}{\textbf{\textit{{#1}}}}}
    
    
    % Define a nice break command that doesn't care if a line doesn't already
    % exist.
    \def\br{\hspace*{\fill} \\* }
    % Math Jax compatability definitions
    \def\gt{>}
    \def\lt{<}
    % Document parameters
    \title{hw1-ijh216}
    
    
    

    % Pygments definitions
    
\makeatletter
\def\PY@reset{\let\PY@it=\relax \let\PY@bf=\relax%
    \let\PY@ul=\relax \let\PY@tc=\relax%
    \let\PY@bc=\relax \let\PY@ff=\relax}
\def\PY@tok#1{\csname PY@tok@#1\endcsname}
\def\PY@toks#1+{\ifx\relax#1\empty\else%
    \PY@tok{#1}\expandafter\PY@toks\fi}
\def\PY@do#1{\PY@bc{\PY@tc{\PY@ul{%
    \PY@it{\PY@bf{\PY@ff{#1}}}}}}}
\def\PY#1#2{\PY@reset\PY@toks#1+\relax+\PY@do{#2}}

\expandafter\def\csname PY@tok@w\endcsname{\def\PY@tc##1{\textcolor[rgb]{0.73,0.73,0.73}{##1}}}
\expandafter\def\csname PY@tok@c\endcsname{\let\PY@it=\textit\def\PY@tc##1{\textcolor[rgb]{0.25,0.50,0.50}{##1}}}
\expandafter\def\csname PY@tok@cp\endcsname{\def\PY@tc##1{\textcolor[rgb]{0.74,0.48,0.00}{##1}}}
\expandafter\def\csname PY@tok@k\endcsname{\let\PY@bf=\textbf\def\PY@tc##1{\textcolor[rgb]{0.00,0.50,0.00}{##1}}}
\expandafter\def\csname PY@tok@kp\endcsname{\def\PY@tc##1{\textcolor[rgb]{0.00,0.50,0.00}{##1}}}
\expandafter\def\csname PY@tok@kt\endcsname{\def\PY@tc##1{\textcolor[rgb]{0.69,0.00,0.25}{##1}}}
\expandafter\def\csname PY@tok@o\endcsname{\def\PY@tc##1{\textcolor[rgb]{0.40,0.40,0.40}{##1}}}
\expandafter\def\csname PY@tok@ow\endcsname{\let\PY@bf=\textbf\def\PY@tc##1{\textcolor[rgb]{0.67,0.13,1.00}{##1}}}
\expandafter\def\csname PY@tok@nb\endcsname{\def\PY@tc##1{\textcolor[rgb]{0.00,0.50,0.00}{##1}}}
\expandafter\def\csname PY@tok@nf\endcsname{\def\PY@tc##1{\textcolor[rgb]{0.00,0.00,1.00}{##1}}}
\expandafter\def\csname PY@tok@nc\endcsname{\let\PY@bf=\textbf\def\PY@tc##1{\textcolor[rgb]{0.00,0.00,1.00}{##1}}}
\expandafter\def\csname PY@tok@nn\endcsname{\let\PY@bf=\textbf\def\PY@tc##1{\textcolor[rgb]{0.00,0.00,1.00}{##1}}}
\expandafter\def\csname PY@tok@ne\endcsname{\let\PY@bf=\textbf\def\PY@tc##1{\textcolor[rgb]{0.82,0.25,0.23}{##1}}}
\expandafter\def\csname PY@tok@nv\endcsname{\def\PY@tc##1{\textcolor[rgb]{0.10,0.09,0.49}{##1}}}
\expandafter\def\csname PY@tok@no\endcsname{\def\PY@tc##1{\textcolor[rgb]{0.53,0.00,0.00}{##1}}}
\expandafter\def\csname PY@tok@nl\endcsname{\def\PY@tc##1{\textcolor[rgb]{0.63,0.63,0.00}{##1}}}
\expandafter\def\csname PY@tok@ni\endcsname{\let\PY@bf=\textbf\def\PY@tc##1{\textcolor[rgb]{0.60,0.60,0.60}{##1}}}
\expandafter\def\csname PY@tok@na\endcsname{\def\PY@tc##1{\textcolor[rgb]{0.49,0.56,0.16}{##1}}}
\expandafter\def\csname PY@tok@nt\endcsname{\let\PY@bf=\textbf\def\PY@tc##1{\textcolor[rgb]{0.00,0.50,0.00}{##1}}}
\expandafter\def\csname PY@tok@nd\endcsname{\def\PY@tc##1{\textcolor[rgb]{0.67,0.13,1.00}{##1}}}
\expandafter\def\csname PY@tok@s\endcsname{\def\PY@tc##1{\textcolor[rgb]{0.73,0.13,0.13}{##1}}}
\expandafter\def\csname PY@tok@sd\endcsname{\let\PY@it=\textit\def\PY@tc##1{\textcolor[rgb]{0.73,0.13,0.13}{##1}}}
\expandafter\def\csname PY@tok@si\endcsname{\let\PY@bf=\textbf\def\PY@tc##1{\textcolor[rgb]{0.73,0.40,0.53}{##1}}}
\expandafter\def\csname PY@tok@se\endcsname{\let\PY@bf=\textbf\def\PY@tc##1{\textcolor[rgb]{0.73,0.40,0.13}{##1}}}
\expandafter\def\csname PY@tok@sr\endcsname{\def\PY@tc##1{\textcolor[rgb]{0.73,0.40,0.53}{##1}}}
\expandafter\def\csname PY@tok@ss\endcsname{\def\PY@tc##1{\textcolor[rgb]{0.10,0.09,0.49}{##1}}}
\expandafter\def\csname PY@tok@sx\endcsname{\def\PY@tc##1{\textcolor[rgb]{0.00,0.50,0.00}{##1}}}
\expandafter\def\csname PY@tok@m\endcsname{\def\PY@tc##1{\textcolor[rgb]{0.40,0.40,0.40}{##1}}}
\expandafter\def\csname PY@tok@gh\endcsname{\let\PY@bf=\textbf\def\PY@tc##1{\textcolor[rgb]{0.00,0.00,0.50}{##1}}}
\expandafter\def\csname PY@tok@gu\endcsname{\let\PY@bf=\textbf\def\PY@tc##1{\textcolor[rgb]{0.50,0.00,0.50}{##1}}}
\expandafter\def\csname PY@tok@gd\endcsname{\def\PY@tc##1{\textcolor[rgb]{0.63,0.00,0.00}{##1}}}
\expandafter\def\csname PY@tok@gi\endcsname{\def\PY@tc##1{\textcolor[rgb]{0.00,0.63,0.00}{##1}}}
\expandafter\def\csname PY@tok@gr\endcsname{\def\PY@tc##1{\textcolor[rgb]{1.00,0.00,0.00}{##1}}}
\expandafter\def\csname PY@tok@ge\endcsname{\let\PY@it=\textit}
\expandafter\def\csname PY@tok@gs\endcsname{\let\PY@bf=\textbf}
\expandafter\def\csname PY@tok@gp\endcsname{\let\PY@bf=\textbf\def\PY@tc##1{\textcolor[rgb]{0.00,0.00,0.50}{##1}}}
\expandafter\def\csname PY@tok@go\endcsname{\def\PY@tc##1{\textcolor[rgb]{0.53,0.53,0.53}{##1}}}
\expandafter\def\csname PY@tok@gt\endcsname{\def\PY@tc##1{\textcolor[rgb]{0.00,0.27,0.87}{##1}}}
\expandafter\def\csname PY@tok@err\endcsname{\def\PY@bc##1{\setlength{\fboxsep}{0pt}\fcolorbox[rgb]{1.00,0.00,0.00}{1,1,1}{\strut ##1}}}
\expandafter\def\csname PY@tok@kc\endcsname{\let\PY@bf=\textbf\def\PY@tc##1{\textcolor[rgb]{0.00,0.50,0.00}{##1}}}
\expandafter\def\csname PY@tok@kd\endcsname{\let\PY@bf=\textbf\def\PY@tc##1{\textcolor[rgb]{0.00,0.50,0.00}{##1}}}
\expandafter\def\csname PY@tok@kn\endcsname{\let\PY@bf=\textbf\def\PY@tc##1{\textcolor[rgb]{0.00,0.50,0.00}{##1}}}
\expandafter\def\csname PY@tok@kr\endcsname{\let\PY@bf=\textbf\def\PY@tc##1{\textcolor[rgb]{0.00,0.50,0.00}{##1}}}
\expandafter\def\csname PY@tok@bp\endcsname{\def\PY@tc##1{\textcolor[rgb]{0.00,0.50,0.00}{##1}}}
\expandafter\def\csname PY@tok@fm\endcsname{\def\PY@tc##1{\textcolor[rgb]{0.00,0.00,1.00}{##1}}}
\expandafter\def\csname PY@tok@vc\endcsname{\def\PY@tc##1{\textcolor[rgb]{0.10,0.09,0.49}{##1}}}
\expandafter\def\csname PY@tok@vg\endcsname{\def\PY@tc##1{\textcolor[rgb]{0.10,0.09,0.49}{##1}}}
\expandafter\def\csname PY@tok@vi\endcsname{\def\PY@tc##1{\textcolor[rgb]{0.10,0.09,0.49}{##1}}}
\expandafter\def\csname PY@tok@vm\endcsname{\def\PY@tc##1{\textcolor[rgb]{0.10,0.09,0.49}{##1}}}
\expandafter\def\csname PY@tok@sa\endcsname{\def\PY@tc##1{\textcolor[rgb]{0.73,0.13,0.13}{##1}}}
\expandafter\def\csname PY@tok@sb\endcsname{\def\PY@tc##1{\textcolor[rgb]{0.73,0.13,0.13}{##1}}}
\expandafter\def\csname PY@tok@sc\endcsname{\def\PY@tc##1{\textcolor[rgb]{0.73,0.13,0.13}{##1}}}
\expandafter\def\csname PY@tok@dl\endcsname{\def\PY@tc##1{\textcolor[rgb]{0.73,0.13,0.13}{##1}}}
\expandafter\def\csname PY@tok@s2\endcsname{\def\PY@tc##1{\textcolor[rgb]{0.73,0.13,0.13}{##1}}}
\expandafter\def\csname PY@tok@sh\endcsname{\def\PY@tc##1{\textcolor[rgb]{0.73,0.13,0.13}{##1}}}
\expandafter\def\csname PY@tok@s1\endcsname{\def\PY@tc##1{\textcolor[rgb]{0.73,0.13,0.13}{##1}}}
\expandafter\def\csname PY@tok@mb\endcsname{\def\PY@tc##1{\textcolor[rgb]{0.40,0.40,0.40}{##1}}}
\expandafter\def\csname PY@tok@mf\endcsname{\def\PY@tc##1{\textcolor[rgb]{0.40,0.40,0.40}{##1}}}
\expandafter\def\csname PY@tok@mh\endcsname{\def\PY@tc##1{\textcolor[rgb]{0.40,0.40,0.40}{##1}}}
\expandafter\def\csname PY@tok@mi\endcsname{\def\PY@tc##1{\textcolor[rgb]{0.40,0.40,0.40}{##1}}}
\expandafter\def\csname PY@tok@il\endcsname{\def\PY@tc##1{\textcolor[rgb]{0.40,0.40,0.40}{##1}}}
\expandafter\def\csname PY@tok@mo\endcsname{\def\PY@tc##1{\textcolor[rgb]{0.40,0.40,0.40}{##1}}}
\expandafter\def\csname PY@tok@ch\endcsname{\let\PY@it=\textit\def\PY@tc##1{\textcolor[rgb]{0.25,0.50,0.50}{##1}}}
\expandafter\def\csname PY@tok@cm\endcsname{\let\PY@it=\textit\def\PY@tc##1{\textcolor[rgb]{0.25,0.50,0.50}{##1}}}
\expandafter\def\csname PY@tok@cpf\endcsname{\let\PY@it=\textit\def\PY@tc##1{\textcolor[rgb]{0.25,0.50,0.50}{##1}}}
\expandafter\def\csname PY@tok@c1\endcsname{\let\PY@it=\textit\def\PY@tc##1{\textcolor[rgb]{0.25,0.50,0.50}{##1}}}
\expandafter\def\csname PY@tok@cs\endcsname{\let\PY@it=\textit\def\PY@tc##1{\textcolor[rgb]{0.25,0.50,0.50}{##1}}}

\def\PYZbs{\char`\\}
\def\PYZus{\char`\_}
\def\PYZob{\char`\{}
\def\PYZcb{\char`\}}
\def\PYZca{\char`\^}
\def\PYZam{\char`\&}
\def\PYZlt{\char`\<}
\def\PYZgt{\char`\>}
\def\PYZsh{\char`\#}
\def\PYZpc{\char`\%}
\def\PYZdl{\char`\$}
\def\PYZhy{\char`\-}
\def\PYZsq{\char`\'}
\def\PYZdq{\char`\"}
\def\PYZti{\char`\~}
% for compatibility with earlier versions
\def\PYZat{@}
\def\PYZlb{[}
\def\PYZrb{]}
\makeatother


    % Exact colors from NB
    \definecolor{incolor}{rgb}{0.0, 0.0, 0.5}
    \definecolor{outcolor}{rgb}{0.545, 0.0, 0.0}



    
    % Prevent overflowing lines due to hard-to-break entities
    \sloppy 
    % Setup hyperref package
    \hypersetup{
      breaklinks=true,  % so long urls are correctly broken across lines
      colorlinks=true,
      urlcolor=urlcolor,
      linkcolor=linkcolor,
      citecolor=citecolor,
      }
    % Slightly bigger margins than the latex defaults
    
    \geometry{verbose,tmargin=1in,bmargin=1in,lmargin=1in,rmargin=1in}
    
    

    \begin{document}
    
    
    \maketitle
    
    

    
    https://github.com/isaachaberman/dsga1011\_hw1.git

    \paragraph{Data Readin}\label{data-readin}

Training data was shuffled and split into training and validation
data-sets of 20,000 and 5,000 respectively. Those functions can be seen
in \texttt{hw\_functions.py} and \texttt{tokenization.py}.

    \subsubsection{Ablation Study Study}\label{ablation-study-study}

\paragraph{Tokenization Scheme}\label{tokenization-scheme}

\begin{longtable}[]{@{}ll@{}}
\toprule
Tokenization Scheme & Accuracy\tabularnewline
\midrule
\endhead
Original & 87.56\tabularnewline
Stop Word Removal & 87.86\tabularnewline
Entities & 87.06\tabularnewline
\bottomrule
\end{longtable}

I tried three tokenization schemes: 1. A variant of the original scheme
used in class. I added an additional removal of html breaks.\\
2. Stop Word Removal. Following the previous scheme, I added a removal
of stop words and apostrophes. 3. Entity Keeping. Using Spacy's
entities, I kept the tokens that had proper entities and removed the
others.

Of the three, the second tokenization scheme performed the best, with a
maximum accuracy of 87.86\%, marginally better than the other maximum
accuracies of 87.56\% and 87.06\%. I also note, that the first and third
tokenization schemes have worse overfitting than the second scheme, with
greater dims in accuracy towards the end of training. I suspect that the
the stop word removal performed the best, as the stop words cloud the
bag of words by being among the most common words and removing them
allows for easier sentiment analysis.

    \paragraph{N-grams}\label{n-grams}

\begin{longtable}[]{@{}ll@{}}
\toprule
N-grams & Accuracy\tabularnewline
\midrule
\endhead
Unigrams & 87.98\tabularnewline
Bigrams & 50.06\tabularnewline
Trigrams & 50.06\tabularnewline
Tetragrams & 50.06\tabularnewline
\bottomrule
\end{longtable}

I tried four n-grams sizes, unigrams, bigrams, trigrams and tetragrams.
For each n-gram scheme, I tested each set of n-grams by themselves. Of
those unigrams performed remarkably better than any of the other schemes
with a validation accuracy of 87.98\% compared to other accuracies of
50\%. I ssupect much of the results may lie in a faulty implementation
or the need for the larger n-grams to be used in conjunction with
unigrams, a configuration I did not test.

    \paragraph{Vocabulary Size}\label{vocabulary-size}

\begin{longtable}[]{@{}ll@{}}
\toprule
Vocabulary Size & Accuracy\tabularnewline
\midrule
\endhead
10k & 88.16\tabularnewline
20k & 88.62\tabularnewline
30k & 88.5\tabularnewline
40k & 88.44\tabularnewline
50k & 88.04\tabularnewline
100k & 88.94\tabularnewline
200k & 88.34\tabularnewline
\bottomrule
\end{longtable}

I varied vocabulary size from 10,000 to 200,000k, with all vocabulary
sizes having similar maximum validation accuracies and similar levels of
overfitting. That being said, I chose a vocabulary size of 100,000 as it
was the best performing. With the removal of stop words and the use of
unigrams, it may be that the vocabulary size is no longer an important
hyperparameter and any nmber of words can be used in the Bag of Words
model.

    \paragraph{Embedding Size}\label{embedding-size}

\begin{longtable}[]{@{}ll@{}}
\toprule
Embedding Size & Accuracy\tabularnewline
\midrule
\endhead
50 & 88.2\tabularnewline
100 & 88.3\tabularnewline
150 & 88.9\tabularnewline
200 & 88.92\tabularnewline
\bottomrule
\end{longtable}

Using the aforementioned hyperparameters, I tested embedding sizes from
50 - 200. All four embeddings performed similarly, with maximum
validation accuracies around 88\% and similar levels of overfitting.
That being said, the best performing embedding size was 200 with a
maximum validation accuracy of 88.92\%. Similarly to the vocabulary
size, its possible the importance of embedding size is lessened by the
prior hyperparameter choices.

    \paragraph{Learning Rate}\label{learning-rate}

\begin{longtable}[]{@{}ll@{}}
\toprule
Learning Rate & Accuracy\tabularnewline
\midrule
\endhead
0.0001 & 80.42\tabularnewline
0.001 & 89.2\tabularnewline
0.01 & 88.62\tabularnewline
0.1 & 86.7\tabularnewline
1 & 86.14\tabularnewline
\bottomrule
\end{longtable}

Unlike the previous two hyperparameters, there were vast differences
between the maximum validation accuracies among the learning rates. The
best learning rate occured at 0.001, one of the smaller learning rates I
tested. From the results it is clear that choosing too small a learning
rate does not produce satisfactory results and the data appears
underfit. And for learning rates that are too large, the model moves to
quickly, and there appear to be poor moves as there is a large dip in
accuracy for both the training and validation accuracy for the biggest
learning rates. Since learning rates can be altered depending on the
epoch, I also tested linear annealing of learning rates. The results can
be seen below.

    \paragraph{Annealing Learning Rate}\label{annealing-learning-rate}

\begin{longtable}[]{@{}ll@{}}
\toprule
Annealing Learning Rate & Accuracy\tabularnewline
\midrule
\endhead
0.0001 & 52.66\tabularnewline
0.001 & 80.62\tabularnewline
0.01 & 88.26\tabularnewline
0.1 & 88.04\tabularnewline
1 & 86.98\tabularnewline
\bottomrule
\end{longtable}

For the linear annealed learning rate, the best results were quite
similar to the best results of the non-annealed learning rate, with
small initial values not improving and the largest initial values
slightly overfitting. Keeping gamma constant, the best performing
annealing learning rate was 0.01 with a maximum validation accuracy of
88.26\%.

    \paragraph{Gamma}\label{gamma}

\begin{longtable}[]{@{}ll@{}}
\toprule
Learning Rate-Gamma & Accuracy\tabularnewline
\midrule
\endhead
0.001 & 88.74\tabularnewline
0.01 & 88.18\tabularnewline
0.1 & 88.9\tabularnewline
1 & 88.24\tabularnewline
\bottomrule
\end{longtable}

For the other annealing learning rate hyperparameter, the best
validation accuracy produced was very similar to the non-annealed
learning rates. For a gamma of 0.1, the validation accuracy was 88.9\%.
Since the optimized annealed learning rate had similar validation
accuracy to the non-annealed learning rate, I chose to use the annealed
learning rate to help prevent overfitting.

    \paragraph{Optimizer}\label{optimizer}

\begin{longtable}[]{@{}ll@{}}
\toprule
Optimizer & Accuracy\tabularnewline
\midrule
\endhead
Adam & 88.48\tabularnewline
SGD & 61.18\tabularnewline
\bottomrule
\end{longtable}

The Adam optimizer produced remarkably better results than the SGD
optimizer. Therefore, I used the Adam optimizer for the final model.
Since the earlier hyperparameter choices had been optimized to Adam, I
suspect that the poor results of SGD may stem from the earlier tuning.

    \subsubsection{Weight Decay}\label{weight-decay}

\begin{longtable}[]{@{}ll@{}}
\toprule
Weight Decay & Accuracy\tabularnewline
\midrule
\endhead
0.0001 & 87.3\tabularnewline
0.001 & 83.34\tabularnewline
0.01 & 54.52\tabularnewline
0.1 & 49.94\tabularnewline
1 & 50.06\tabularnewline
\bottomrule
\end{longtable}

Lastly, I did a hyperparameter search over the weight decay to help
prevent overfitting in the final model. Of the weight decay options,
0.0001 produced the best validation accuracy with a value of 87.3\%.
Given too large a weight decay, the model performs rather poorly as the
constraint of the weight decay destroys performance gains, while the
smaller weight decays produce better results since they restrict the
model less. Since there does not appear to be much overfitting in the
final results, the validation accuracy not dipping, the smallest weight
decay was an appropriate choice.

    \paragraph{Final Model}\label{final-model}

Using the best performing hyperparameters, the model produces a final
test accuracy of 86.7\% compared to a test accuracy of 82.61\% using the
non-ablated hyperparameters. In other words, by using a doing an
abalation study, we were able to successfully classify an additional
1000 reviews.

I suspect with some additional tuning we may be able to improve the test
accuracy. I did not test the rate in which the linear annealing occurs
as a function of the step size, nor did I test the use of unigrams and
bigrams together, two other hyperparameter choices that may have
improves results.

    \paragraph{Misclassified Examples}\label{misclassified-examples}

'two things changed first kid anymore second new seagal movies terrible
this opinion worst movie derailed starred jean claude van damme there
plot movie plot excuse shoot terrible action scenes painful watch i love
action movies action movie this nt movie group irritating scenes
connected annoying way kill viewers love action movies i sorry nt
respect liked movie he intelligence problem i hope seagal new good
movies future good luck' - While the tokenization removes some of the
readability, the reviews can still be analyzed. For instance, this first
misclassification might have been caused by the repeated use of "love" a
term associated with positive sentiment but here is used as background.
The author loves action films, yet does not like this film.

'after watch movie i surprised like i think worst comedy i seen if think
seen worst comedy wait watch crap not funny ok laugh maybe high acting
terrible story nt exist even like i hate movie ok ok nice babes nice
tricks beautiful mountains s watch porno set mountains instead acting
story line better sometimes imdb i read bad review movie end awful
believe pure trash do nt waste time money' - This review refers to
"favourite movies" (it might be worth standardizing spellings of words
in a different tokenization scheme), "positive", "brightest" and other
positive words, which in conjunction with other words would produce
negative sentiment, but as seen by themselves produce positive
sentiment.

'after watch movie i surprised like i think worst comedy i seen if think
seen worst comedy wait watch crap not funny ok laugh maybe high acting
terrible story nt exist even like i hate movie ok ok nice babes nice
tricks beautiful mountains s watch porno set mountains instead acting
story line better sometimes imdb i read bad review movie end awful
believe pure trash do nt waste time money' - The reviewer spends much of
the review trashing the movie, yet starts the review with "i surprised
like," i.e. the reviewer enjoyed the film even though there were
negative elements to it. Therefore, the review was classified negatively
even though it was a positive review.

    \paragraph{Classified Examples}\label{classified-examples}

'after astronaut dies space brought military base inside man discovered
alien embryos host terrible alien invasion this film comes director
bernard l. kowalski directed attack giant leeches separate review better
known work columbo executive producer roger corman known creator better
films particularly 1960s this movie cheesy poorly constructed what comes
interesting poor effects actual film one scene shows close alien embryos
embarrassing cartoon representation even 1958 and grown alien appears ll
wonder wearing shoes or perceptive ll wonder ve seen alien suit movies
by means worst science fiction film ll and credit alien host overtaking
military base idea predates the thing alien number years i nt know films
inspired way i doubt ahead time beyond film flops great heckling
drinking i nt seen mystery science theater version sure film worthy
insults' - There is clear negative sentiment in this review with words
like "flops", "embarassing", "worst", and "insults."

'minor spoilersin chicago grace beasley kathy bates housewife having
years marriage lawyer max beasley dan aykroyd hysterical psychotic dwarf
daughter law meredith eaton grace worships singer victor fox jonathan
price present tv chicago spots row tv promotion kate calls wins ticket
max simultaneously asks divorce claiming lives monotonous grace
depressed goes audience informed chicago serial killer uses crossbow
killed victor fox with broken heart decides fly england victor foxs
funeral there realizes gay friend mate dirk simpson rupert everett they
fly chicago trying find killer this movie delightful original weird
dramatic comedy having bizarre characters it huge potential cult movie
presences julie andrews barry joke nicolas roegs masterpiece do nt look
now wears red raincoat chicagos underground city the beginning movie
jonathan price singing hitchcock railway wonderful i repeated
consecutive times the cast magnificent performance highlighting kathy
bates jonathan price rupert everett unknown meredith eaton indeed movie
excellent entertainment my vote title brazil amor love proof' - The
reviewer refers to this film as "delightful" a positively charged word.
Since much of the review is a summary of the film, there is littler
other sentiment.

'this movie potential good movie eyes nicholas sparks great romance
author movie chance great the notebook s sets apart notebook dream team
leads mcadams gosling balance thrown miserably inept acting channing
felt lot scenes uneven purely performance lot emotion scenes lost nt act
leaving awkward uneven situation amanda given great performance tatum
drop ball mood lost scene nt recover this story deserved cast right got
pretty boy nt act tatum stick s good movies physical ability albeit
horrible like gi joe step fighting the talks better try think jaded
hater channing tatum i went movie open mind i ve surprised time likes
adam sandler reign me i gave chance tatum i nt view sum past roles
purely performance movie sadly letdown' - While the reviewer has qualms
with the film, they enjoy the lead actressses performance and Nicholas
Sparks novels and therefore the positive sentiment outweighs the
negative.


    % Add a bibliography block to the postdoc
    
    
    
    \end{document}
